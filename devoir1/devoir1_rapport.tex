%Critique.tex
%Par Guillaume Lahaie
%LAHG04077707
%
%Critique du SRS Gestion de centre sportif fourni dans le cours. J'utilise un gabarit de tex pour l'écriture

%%%%%%%%%%%%%%%%%%%%%%%%%%%%%%%%%%%%%%%%%
% Simple Sectioned Essay Template
% LaTeX Template
%
% This template has been downloaded from:
% http://www.latextemplates.com
%
% Note:
% The \lipsum[#] commands throughout this template generate dummy text
% to fill the template out. These commands should all be removed when 
% writing essay content.
%
%%%%%%%%%%%%%%%%%%%%%%%%%%%%%%%%%%%%%%%%%

%----------------------------------------------------------------------------------------
%	PACKAGES AND OTHER DOCUMENT CONFIGURATIONS
%----------------------------------------------------------------------------------------

\documentclass[11pt]{article} % Default font size is 12pt, it can be changed here
\renewcommand{\familydefault}{\rmdefault}

%Pour l'encodage avec accents
\usepackage[utf8]{inputenc}

\usepackage{helvet}
\renewcommand{\familydefault}{\sfdefault}

\usepackage{afterpage}

\usepackage[left=2.2cm,top=2.2cm,right=2.2cm,bottom=2.2cm,nohead]{geometry} % Required to change the page size to A4
\geometry{letterpaper} % Set the page size to be A4 as opposed to the default US Letter

\usepackage{graphicx} % Required for including pictures

\usepackage{float} % Allows putting an [H] in \begin{figure} to specify the exact location of the figure
\usepackage{wrapfig} % Allows in-line images such as the example fish picture

\usepackage{lipsum} % Used for inserting dummy 'Lorem ipsum' text into the template


\linespread{1.2} % Line spacing

%\setlength\parindent{0pt} % Uncomment to remove all indentation from paragraphs

\graphicspath{{./Pictures/}} % Specifies the directory where pictures are stored
\usepackage[french,english]{babel}

%Comportement d'un paragraphe
\setlength{\parskip}{\baselineskip}%
\setlength{\parindent}{0pt}%

%Widows/orphans
\widowpenalty10000
\clubpenalty10000

\usepackage[hidelinks]{hyperref}

%Meta-info
\title{INM5151 - Autoévaluation}
\author{Guillaume Lahaie}
\date{Remise: 23 juillet 2013}

\hypersetup{
  pdftitle={INM5151 - Rapport d'autoévaluation},
  pdfauthor={Guillaume Lahaie}
}

\newcommand\blankpage{%
  \null
  \thispagestyle{empty}%
  \addtocounter{page}{-1}%
  \newpage}

\begin{document}
\selectlanguage{french}

%----------------------------------------------------------------------------------------
%	TITLE PAGE
%----------------------------------------------------------------------------------------

\begin{titlepage}

\newcommand{\HRule}{\rule{\linewidth}{0.5mm}} % Defines a new command for the horizontal lines, change thickness here

\center % Center everything on the page

\textsc{\LARGE Université du Québec à Montréal}\\[1.5cm] % Name of your university/college
\textsc{\Large INM5151}\\[0.5cm] % Major heading such as course name

\HRule \\[1.5cm]
{ \huge \bfseries Rapport d'autoévaluation}\\[0.4cm] % Title of your document
\HRule \\[1.5cm]

\begin{minipage}{0.4\textwidth}
\begin{flushleft} \large
\emph{Par:}\\
Guillaume Lahaie \\ LAHG04077707 % Your name
\end{flushleft}
\end{minipage}
~
\begin{minipage}{0.4\textwidth}
\begin{flushright} \large
\emph{Remis à:} \\
Jacques Berger % Supervisor's Name
\end{flushright}
\end{minipage}\\[4cm]

{\large \emph{Date de remise:} \\ Le 23 juillet 2013}\\[3cm] % Date, change the \today to a set date if you want to be precise

%\includegraphics{Logo}\\[1cm] % Include a department/university logo - this will require the graphicx package

\vfill % Fill the rest of the page with whitespace

\end{titlepage}
\blankpage

%----------------------------------------------------------------------------------------
%	TABLE OF CONTENTS
%----------------------------------------------------------------------------------------

\tableofcontents % Include a table of contents

\newpage % Begins the essay on a new page instead of on the same page as the table of contents 

%%----------------------------------------------------------------------------------------
% INTRODUCTION
%----------------------------------------------------------------------------------------
 
\section{Autoévaluation} % Major section

{\LARGE Note: 95 / 100}

En regardant les suivis de chaque semaine, je crois que j'ai bien participer à toutes les étapes du projet, toutefois j'ai travaillé
sur le prototype seulement en fin de projet, j'aurais probablement pu m'impliquer un peu plus tôt. En contrepartie, je crois que j'ai
participé un peu plus à la rédaction et l'édition du SRS. 

À certains moments durant le projet, je crois que j'aurais dû être un peu plus claire dans mes communications, et répondre aux courriels
plus rapidement. J'aurais aussi du tenir compte de l'importance du prototype un peu plus.


%----------------------------------------------------------------------------------------
% SECTIONS DU DOCUMENT
%----------------------------------------------------------------------------------------
 
\section{Éric Fournier (chef d'équipe)}

{\LARGE Note: 100 / 100}
 
Rien à redire sur son travail. A pris en charge le SRS au début de la session, et ensuite a
fait une bonne partie du prototype, en s'occupant aussi des suivis hebdomadaires, et aussi
faire le suivi avec chaque membre.
 
%----------------------------------------------------------------------------------------
% SECTION VALIDATION
%----------------------------------------------------------------------------------------
 
\section{Maxime Girard}

{\LARGE Note: 90 / 100}

Il s'est occupé d'une des parties importantes du prototype surtout. Je pense qu'il aurait pu nous donner un
peu plus de feedback sur ses progrès, généralement j'ai l'impression que c'était surtout ça avance, ça avance pas, 
pas plus de détail que ça.


%----------------------------------------------------------------------------------------
% CONCLUSION
%----------------------------------------------------------------------------------------
 
\section{Marco Gagliano} 
 
{\LARGE Note: 80 / 100}

Malgré qu'il ait participé à toutes les réunions et qu'il ait faite une grande partie du protoype, c'était
difficile de prévoir quand le travail serait fait. Il y a eu quelques semaines où il n'a rien fait. Au moins,
à une reprise, il nous en a averti à l'avance. À certains moments, il disait n'avoir plus le temps
de faire certains parties du projet, mais finalement il les faisait, c'était donc un peu difficile
d'organiser les tâches.

J'ai trouvé aussi qu'on avait des attentes différentes à propos du projet. Il a dit à quelques reprises
que ce n'était pas nécessaire de trop se forcer, que si on avait un B+ pour le cours c'était assez. J'avais
une opinion différente.
 
\end{document}